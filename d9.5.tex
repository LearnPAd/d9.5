%%% LearnPAd Document template / example using learnpad.cls class for styling
%%% 20140704, 
%%% Guglielmo De Angelis <guglielmo.deangelis@isti.cnr.it>
%%% Andrea Polini <andrea.polini@unicam.it>

\documentclass{learnpad}

%%% ------------------------------------------------------
%%% ---------------- The Title
%%% ------------------------------------------------------
\title{Technology-oriented Learn PAd whitepaper}

%%% ------------------------------------------------------
%%% ---------------- The Sub-Title
%%% ------------------------------------------------------
\subtitle{First version}

%%% ------------------------------------------------------
%%% ---------------- The Name of the Deliverable
%%% ------------------------------------------------------
\deliverableno{D9.5}

%%% ------------------------------------------------------
%%% ---------------- The Authors
%%% ------------------------------------------------------
\authors{
Jean Simard (XWIKI)
}

%%% ------------------------------------------------------
%%% ---------------- The Editors
%%% ------------------------------------------------------
\editors{Jean Simard}

%%% ------------------------------------------------------
%%% ---------------- The reviewers
%%% ------------------------------------------------------
\reviewers{Andrea Polini (UNICAM)}

%%% ------------------------------------------------------
%%% ---------------- The date
%%% ------------------------------------------------------
\date{\today}

%%% ------------------------------------------------------
%%% ---------------- deliverable info
%%% ---------------- choose among : Report / Other / Prototype
%%% ------------------------------------------------------
\naturedeliverable{Report}%
%%% ------------------------------------------------------
%%% ---------------- deliverable dissemination levele
%%% ---------------- choose among the two options below:
\disseminationlevelpublic
% \disseminationlevelconfidential
%%% ------------------------------------------------------
\version{1.0}%
\contractualdeliverydate{24 January 2016}%
\actualdeliverydate{24 January 2016}%
\contributingwp{WP9}%

%%% ------------------------------------------------------
%%% ---------------- abstract
%%% ------------------------------------------------------

\abstract{Provides a technology oriented overview of Learn PAd that is targeted
to readers from research and software vendor communities with a goal to build
interest in adopting Learn PAd concepts in further research and integrating
technological components in external e-learning, modelling and middleware
software applications.}

%%% ------------------------------------------------------
%%% ---------------- Keywords
%%% ------------------------------------------------------
\keywords{platform, white paper}

%%% ------------------------------------------------------
%%% ---------------- review table
%%% ------------------------------------------------------
\reviewoutline{2 Dec. 2015}{0.1}{}{}
\reviewdraft{27 Dec. 2015}{1.0}{}{}
\reviewinternal{10 Jan. 2016}{2.0}{Andrea Polini, Antonia Bertolino}{}
\reviewcandidatefinal{24 Jan. 2016}{3.0}{Antonia Bertolino}{}

\begin{document}

\frontmatter
\maketitle

%% ------------------------------------------------------
%% ---------------- document history
%% ------------------------------------------------------
\begin{history}
  \historyitem{0.1}{ToC}{Jean Simard} 
\end{history}

%%% ------------------------------------------------------
%%% ---------------- review table with the previous info
%%% ------------------------------------------------------
\reviewtable

% %%% ------------------------------------------------------
% %%% ---------------- acronyms
% %%% ------------------------------------------------------
% \begin{acronyms}
%   \acronym{CA}{Consortium Agreement}%
%   \acronym{DL}{Deliverable Leader}%
%   \acronym{DOW}{Description of Work}%
%   \acronym{EC}{European Commission}%
%   \acronym{EL}{Exploitation Leader}%
%   \acronym{GA}{Grant Agreement}%
%   \acronym{IPR}{Intellectual Property Rights}%
%   \acronym{PAB}{Project Advisory Board}%
%   \acronym{PCB}{Project Coordination Board}%
%   \acronym{PL}{Project Leader}%
%   \acronym{PMB}{Project Management Board}%
%   \acronym{PO}{Project Officer}%
%   \acronym{SL}{Scientific Leader}%
%   \acronym{S\&T}{Scientific and Technical}%
%   \acronym{TL}{Technical Leader}%
%   \acronym{WP}{Work Package}%
%   \acronym{WPL}{Work Package Leader}
% %   \acronym{\dots}{\dots~\dots}%
% \end{acronyms}

\tableofcontents

%%% ------------------------------------------------------
% In case you don't need one of the following list 
% just comment the line
%%% ------------------------------------------------------

% \listoftables 
% \listoffigures 
% \listoflistings

%%% ------------------------------------------------------

\mainmatter

%%% ------------------------------------------------------
%%% ---------------- Start whit chapter and sections here!
%%% ------------------------------------------------------

\chapter{Introduction about e-Learning in Public Administration}
\label{ch:intro}
Some kind of simplified State-of-the-art about existing technologies on these topics.

\chapter{Why Learn PAd?}
\label{ch:problematic}
What are the problems of existing tools and what Learn PAd is trying to address.

\chapter{What is Learn PAd?}
\label{ch:platform}
Macro description of Learn PAd platform, what are the elements, the workflow in it and the technologies used.

\chapter{Partners}
\label{ch:partners}
List of partners with a small paragraph that explain domain of expertise.

% ---------------------------------- Start whit annexes here!
% ----------------------------------

% \annex{}

% ---------------------------------- Start EndNotes here!  
% ---------------------------------- 

% Plese use this command if and only if your text includes endnotes.
% Otherwise, comment it.

% \theendnotes

% ---------------------------------- Bibliography starts here
% ----------------------------------

\bibliographystyle{plain}
\bibliography{biblio}

\end{document}
