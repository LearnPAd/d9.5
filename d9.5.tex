%%% LearnPAd Document template / example using learnpad.cls class for styling
%%% 20140704, 
%%% Guglielmo De Angelis <guglielmo.deangelis@isti.cnr.it>
%%% Andrea Polini <andrea.polini@unicam.it>

\documentclass{learnpad}

%%% ------------------------------------------------------
%%% ---------------- The Title
%%% ------------------------------------------------------
\title{Technology-oriented Learn PAd whitepaper}

%%% ------------------------------------------------------
%%% ---------------- The Sub-Title
%%% ------------------------------------------------------
\subtitle{First version}

%%% ------------------------------------------------------
%%% ---------------- The Name of the Deliverable
%%% ------------------------------------------------------
\deliverableno{D9.5}

%%% ------------------------------------------------------
%%% ---------------- The Authors
%%% ------------------------------------------------------
\authors{
Jean Simard (XWIKI)
}

%%% ------------------------------------------------------
%%% ---------------- The Editors
%%% ------------------------------------------------------
\editors{Jean Simard}

%%% ------------------------------------------------------
%%% ---------------- The reviewers
%%% ------------------------------------------------------
\reviewers{Andrea Polini (UNICAM)}

%%% ------------------------------------------------------
%%% ---------------- The date
%%% ------------------------------------------------------
\date{\today}

%%% ------------------------------------------------------
%%% ---------------- deliverable info
%%% ---------------- choose among : Report / Other / Prototype
%%% ------------------------------------------------------
\naturedeliverable{Report}%
%%% ------------------------------------------------------
%%% ---------------- deliverable dissemination levele
%%% ---------------- choose among the two options below:
\disseminationlevelpublic
% \disseminationlevelconfidential
%%% ------------------------------------------------------
\version{0.2}%
\contractualdeliverydate{24 January 2016}%
\actualdeliverydate{24 January 2016}%
\contributingwp{WP9}%

%%% ------------------------------------------------------
%%% ---------------- abstract
%%% ------------------------------------------------------

\abstract{Provides a technology oriented overview of Learn PAd that is targeted
to readers from research and software vendor communities with a goal to build
interest in adopting Learn PAd concepts in further research and integrating
technological components in external e-learning, modelling and middleware
software applications.}

%%% ------------------------------------------------------
%%% ---------------- Keywords
%%% ------------------------------------------------------
\keywords{platform, white paper}

%%% ------------------------------------------------------
%%% ---------------- review table
%%% ------------------------------------------------------
\reviewoutline{2 Dec. 2015}{0.2}{}{}
\reviewdraft{27 Dec. 2015}{1.0}{}{}
\reviewinternal{10 Jan. 2016}{2.0}{Andrea Polini, Antonia Bertolino}{}
\reviewcandidatefinal{24 Jan. 2016}{3.0}{Antonia Bertolino}{}

\begin{document}

\frontmatter
\maketitle

%% ------------------------------------------------------
%% ---------------- document history
%% ------------------------------------------------------
\begin{history}
  \historyitem{0.1}{ToC}{Jean Simard} 
  \historyitem{0.2}{ToC}{Jean Simard} 
\end{history}

%%% ------------------------------------------------------
%%% ---------------- review table with the previous info
%%% ------------------------------------------------------
\reviewtable

% %%% ------------------------------------------------------
% %%% ---------------- acronyms
% %%% ------------------------------------------------------
% \begin{acronyms}
%   \acronym{CA}{Consortium Agreement}%
%   \acronym{DL}{Deliverable Leader}%
%   \acronym{DOW}{Description of Work}%
%   \acronym{EC}{European Commission}%
%   \acronym{EL}{Exploitation Leader}%
%   \acronym{GA}{Grant Agreement}%
%   \acronym{IPR}{Intellectual Property Rights}%
%   \acronym{PAB}{Project Advisory Board}%
%   \acronym{PCB}{Project Coordination Board}%
%   \acronym{PL}{Project Leader}%
%   \acronym{PMB}{Project Management Board}%
%   \acronym{PO}{Project Officer}%
%   \acronym{SL}{Scientific Leader}%
%   \acronym{S\&T}{Scientific and Technical}%
%   \acronym{TL}{Technical Leader}%
%   \acronym{WP}{Work Package}%
%   \acronym{WPL}{Work Package Leader}
% %   \acronym{\dots}{\dots~\dots}%
% \end{acronyms}

\tableofcontents

%%% ------------------------------------------------------
% In case you don't need one of the following list 
% just comment the line
%%% ------------------------------------------------------

% \listoftables 
% \listoffigures 
% \listoflistings

%%% ------------------------------------------------------

\mainmatter

%%% ------------------------------------------------------
%%% ---------------- Start with chapter and sections here!
%%% ------------------------------------------------------

\chapter{E-Learning and Knowledge Management of processes in Public Administration}
\label{ch:intro}
Draw the context. Half a page.

\chapter{Public Administration is already using it}
\label{ch:sota}
State-of-the-art about existing technologies on these topics.
Half a page/a page.

\chapter{Needed features for adoption in Public Administration}
\label{ch:problematic}
What are the problems of existing tools and what Learn PAd is trying to address.
\begin{itemize}
	\item new-comers
	\item learning of processes
	\item centralization of the documentation
	\item continuous and collaborative maintenance of the knowledge
\end{itemize}

\chapter{Learn PAd platform: solutions and technologies}
\label{ch:platform}
Macro description of Learn PAd platform, what are the elements, the workflow in it and the technologies used.
Need some graphic (UML?) to describe the platform.

Here is a list of technologies we should address:
\begin{itemize}
	\item EMF: transformations of model
	\item Adoxx/MagicDraw: Modeling tools
	\item Ontology: Recommendation capability
	\item Model verification: PetriNET?
	\item Content Analysis: have to ask about the technology
\end{itemize}
The goal is not to enter to much details but explains in which context and for what feature these technologies are used, then mention the name of the technologie and the specificities (benefit/drawbacks) of it.

\chapter{Partners}
\label{ch:partners}
List of partners with a small paragraph that explain domain of expertise.

% ---------------------------------- Start with annexes here!
% ----------------------------------

% \annex{}

% ---------------------------------- Start EndNotes here!  
% ---------------------------------- 

% Plese use this command if and only if your text includes endnotes.
% Otherwise, comment it.

% \theendnotes

% ---------------------------------- Bibliography starts here
% ----------------------------------

\bibliographystyle{plain}
\bibliography{biblio}

\end{document}
